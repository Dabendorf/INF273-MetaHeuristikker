%%% LaTeX Template originaly created by Karol Kozioł (mail@karol-koziol.net) and modified for ShareLaTeX use

\documentclass[a4paper,11pt]{article}

\usepackage[T1]{fontenc}
\usepackage[utf8]{inputenc}
\usepackage{graphicx}
\usepackage[british]{babel}
\usepackage[german=quotes]{csquotes}
\usepackage[dvipsnames]{xcolor}

%\renewcommand\familydefault{\sfdefault}
%\usepackage{tgheros}
%\usepackage[defaultmono]{droidmono}

\usepackage{amsmath,amssymb,amsthm,textcomp}
\usepackage{enumerate}
\usepackage{multicol}
\usepackage{tikz}

\usepackage{geometry}
\geometry{left=25mm,right=25mm,%
bindingoffset=0mm, top=20mm,bottom=20mm}
\usepackage{graphicx}
%\usepackage{subcaption}
%\usepackage{mwe}


\linespread{1.3}

\newcommand{\linia}{\rule{\linewidth}{0.5pt}}

% custom theorems if needed
\newtheoremstyle{mytheor}
    {1ex}{1ex}{\normalfont}{0pt}{\scshape}{.}{1ex}
    {{\thmname{#1 }}{\thmnumber{#2}}{\thmnote{ (#3)}}}

\theoremstyle{mytheor}
\newtheorem{defi}{Definition}

% my own titles
\makeatletter
\renewcommand{\maketitle}{
\begin{center}
\vspace{2ex}
{\huge \textsc{\@title}}
\vspace{1ex}
\\
\linia\\
\@author \hfill \@date
\vspace{4ex}
\end{center}
}
\makeatother
%%%

% custom footers and headers
\usepackage{fancyhdr}
\pagestyle{fancy}
\lhead{}
\chead{}
\rhead{}
%\lfoot{Assignment \textnumero{} 5}
\cfoot{}
\rfoot{Page \thepage}
\renewcommand{\headrulewidth}{0pt}
\renewcommand{\footrulewidth}{0pt}
%

% code listing settings
\usepackage{listings}
\lstset{
    language=Python,
    basicstyle=\ttfamily\small,
    aboveskip={1.0\baselineskip},
    belowskip={1.0\baselineskip},
    columns=fixed,
    extendedchars=true,
    breaklines=true,
    tabsize=4,
    %prebreak=\raisebox{0ex}[0ex][0ex]{\ensuremath{\hookleftarrow}},
    prebreak=,
    frame=lines,
    showtabs=false,
    showspaces=false,
    showstringspaces=false,
    keywordstyle=\color[rgb]{0.627,0.126,0.941},
    commentstyle=\color[rgb]{0.133,0.545,0.133},
    stringstyle=\color[rgb]{01,0,0},
    numbers=left,
    numberstyle=\small,
    stepnumber=1,
    numbersep=10pt,
    captionpos=t,
    escapeinside={\%*}{*)}
}

% ref packages
\usepackage{nameref}
% folowing  must be in this order
\usepackage{varioref}
\usepackage{hyperref}
\usepackage{cleveref}

\setlength\parindent{0pt}

%%%----------%%%----------%%%----------%%%----------%%%

\begin{document}

\title{INF273 – Assignment 4}

\author{Lukas Schramm}

\date{\today}

\maketitle

\section{Functions}
Currently, I am maintaining the following helper functions in my Utils file.\footnote{The green ones are changes or additions from the last assignment}
\begin{itemize}
\item load\_problem: Given a path to a file, it reads the content of the file into a dictionary of information.
\item feasibility\_check: It takes a solution (list) and a problem dictionary and checks if the solution is feasible. If it is not feasible, it outputs the reason why. It does not check validity. I hope it works correctly
\item cost\_function: It takes a solution (list) and a problem dictionary and calculates the cost of the function. As feasibility\_check it does not check if the original solution was valid.
\item splitting\_a\_list\_at\_zeros: Helper function which splits a solution into vehicles and if needed a dummy vehicle.
\item initial\_solution: Generates an initial default solution to start with. This is always a solution where the dummy vehicle handles all calls.

\item random\_solution: Generates a random solution. The generator itself is quite bad in my view because I overtuned it a bit. It automatically gives one vehicle exactly one call and the rest goes to the dummy vehicle. That way I got solutions for file 3 and 4 but the solutions for all files are quite bad.\footnote{But since we do not need that random solution generator any longer I keep it like that.}
\item blind\_random\_search: Takes a problem and a number of iterations to find the best out of n random feasible solutions if any is found.
\item blind\_search\_latex\_generator: This function runs the blind\_random\_search and writes the data into \LaTeX tables since I am obviously too lazy to do it myself.
\item latex\_add\_line: Adds a new result line into an results table of this file.
\item latex\_replace\_line: Change the optimal solution and its seed in that file.

\item \textcolor{ForestGreen}{feasibility\_helper: }A function which works similar to the feasibility helper but calculates the feasibility only for one vehicle.
\item \textcolor{ForestGreen}{cost\_helper: }A function which calculates the cost only for one vehicle.
\item \textcolor{ForestGreen}{cost\_helper\_transport\_only: }A function which calculates the cost only for one vehicle, but also only the transporting cost. This is perfect for adding a call at different positions and calculating the cost.
\item \textcolor{ForestGreen}{greedy\_insert\_one\_call\_one\_vehicle: }A helper function which greedily finds the best insertion position for a call for one vehicle.
\item \textcolor{ForestGreen}{helper\_regretk\_insert\_one\_call\_vehicle: }A helper function which calculates the regret-k values for all valid insertion positions of a call into a vehicle.

The following six functions three removal and insertion heuristics which then are combined to my full neighbouring functions.
\item \textcolor{ForestGreen}{remove\_random\_call: }Removes $n$ random calls from it's current position.
\item \textcolor{ForestGreen}{remove\_highest\_cost\_call: } Removes the $n$ calls with the highest cost from its current position. This function has a random choice function based on the exponential probability distribution which chooses calls randomly, but weighted by how costly they are.
\item \textcolor{ForestGreen}{remove\_dummy\_call: } Removes $n$ calls from the dummy vehicle.
\item \textcolor{ForestGreen}{insert\_regretk: }This function performs regret-k and inserts $n$ calls into it. This function does not yet have a random functionality, which probability will be added.
\item \textcolor{ForestGreen}{insert\_greedy: }Performs greedy insert for $n$ calls. This function does not yet have a random functionality, which probability will be added.
\item \textcolor{ForestGreen}{insert\_back\_to\_dummy: }It takes $n$ calls and puts them back into the dummy vehicle.
\end{itemize}\medskip

\textbf{Background information: }I have been working on these neighbouring functions for eight weeks but absolutely nothing was working. After a long debugging process with Morten, I have decided to delete my entire code and rewrite it. This was a stressful but great decision and finally, these functions above are working. There are in fact the first working functions. I am planning in making them better and maybe adding one or two, but for now, this is my solution. Also, the running time is terrible and must be improved. The reported times are actually times from the high-performance university server in Berlin and not from my ancient notebook, so there might be a bigger scaling factor existing. For assignment 5, I will be reporting the correct times.\footnote{I guess}\medskip

My three neighbouring functions are the following three, combining pairs of the above removal and insertion functions.
\begin{itemize}
\item Steven: Removing $n$ random calls and inserting those greedily.
\item Jackie: Removing $n$ highest cost calls and inserting them with regretk.
\item Sebastian: Removing $n$ dummy calls and inserting them greedily.
\end{itemize}

\medskip
If there are any questions or nice recommendations to get a better structure, just send me a message.

\clearpage
Moreover, there is a new file for Heuristics where I collect all of the important algorithms and their helper functions.
\begin{itemize}
\item alter\_solution\_1insert: A function which takes a current solution and outputs a next solution by using the 1-insert-operation. The output is not necessary feasible, but of course valid
\item alter\_solution\_2exchange: A function which takes a current solution and outputs a next solution by using the 2-exchange-operation. The output is not necessary feasible, but of course valid
\item alter\_solution\_3exchange: A function which takes a current solution and outputs a next solution by using the 3-exchange-operation. The output is not necessary feasible, but of course valid
\item \textcolor{ForestGreen}{alter\_solution\_steven: }Neighbour function, see above.
\item \textcolor{ForestGreen}{alter\_solution\_jackie: }Neighbour function, see above.
\item \textcolor{ForestGreen}{alter\_solution\_sebastian: }Neighbour function, see above.


\item \textcolor{ForestGreen}{local\_search: This function takes a problem, an initial solution, a number of iterations (10.000) and the allowed neighbouring function and performs a local search}
\item \textcolor{ForestGreen}{simulated\_annealing: This function takes a problem, an initial solution, a number of iterations (10.000) and the allowed neighbouring function and performs a simulated annealing}
\item \textcolor{ForestGreen}{local\_search\_sim\_annealing\_latex: This function takes as input the allowed neighbouring function(s), the heuristics method, the problem and the number of iterations and performs the heuristics on randomly chosen seeds. It then calculates the average time and objective and runs the \LaTeX functions to change the tables of this PDF}
\end{itemize}

\clearpage

\section{Result tables}
\begin{table}[ht]
\centering
\caption{Call\_7\_Vehicle\_3}
\label{tab:call7vehicle3}
\begin{tabular}{|r|r|r|r|r|}
Method & Average objective & Best objective & Improvement (\%) & Running time \\
\hline
Random search & 2289893.35 & 2120884 & 34.59\% & 0.62s\\
Local Search-1-insert & 1416012.10 & 1134176 & 65.02\% & 0.96s\\
Local Search-2-exchange & 1243141.90 & 1134176 & 65.02\% & 0.99s\\
Local Search-3-exchange & 1238358.70 & 1134176 & 65.02\% & 0.88s\\
Simulated Annealing-1-insert & 1301446.80 & 1134176 & 65.02\% & 1.16s\\
Simulated Annealing-2-exchange & 1314847.60 & 1134176 & 65.02\% & 0.79s\\
Simulated Annealing-3-exchange & 1240012.20 & 1134176 & 65.02\% & 0.74s\\
Simulated Annealing & 3242625.00 & 3242625 & 0.00\% & 2.07s\\
Simulated Annealing & 1427187.10 & 1153343 & 64.43\% & 2.16s\\
Simulated Annealing & 1332161.00 & 1134176 & 65.02\% & 4.05s\\
\end{tabular}%Call\_7\_Vehicle\_3
\end{table}
\begin{lstlisting}[label={lst:call7vehicle3},caption=Optimal solution call\_7\_vehicle\_3]
sol = [4, 4, 7, 7, 0, 2, 2, 0, 1, 5, 5, 3, 3, 1, 0, 6, 6]
seeds = [514451821, 307419318, 623032474, 625312418, 249696821, 199931173, 413312881, 459875172, 763585513, 598106986]
\end{lstlisting}%Call\_7\_Vehicle\_3
\clearpage


\begin{table}[ht]
\centering
\caption{Call\_18\_Vehicle\_5}
\label{tab:call18vehicle5}
\begin{tabular}{|r|r|r|r|r|}
Method & Average objective & Best objective & Improvement (\%) & Running time \\
\hline
Random search & 7195792.08 & 6215552 & 29.80\% & 0.80s\\
Local Search-1-insert & 2985468.70 & 2535568 & 71.70\% & 1.33s\\
Local Search-2-exchange & 3114722.90 & 2512517 & 71.96\% & 1.17s\\
Local Search-3-exchange & 3380136.20 & 2878735 & 67.87\% & 1.11s\\
Simulated Annealing-1-insert & 3064756.80 & 2668336 & 70.22\% & 1.63s\\
Simulated Annealing-2-exchange & 2911592.60 & 2525599 & 71.81\% & 1.34s\\
Simulated Annealing-3-exchange & 3124333.50 & 2835545 & 68.35\% & 1.08s\\
Simulated Annealing & 3624341.50 & 3510240 & 60.82\% & 7.63s\\
\end{tabular}%Call\_18\_Vehicle\_5
\end{table}
\begin{lstlisting}[label={lst:call18vehicle5},caption=Optimal solution call\_18\_vehicle\_5]
sol = [4, 4, 15, 15, 11, 12, 11, 12, 0, 6, 6, 5, 5, 17, 17, 16, 16, 0, 8, 18, 18, 8, 13, 13, 0, 7, 7, 3, 3, 10, 1, 10, 1, 0, 9, 9, 14, 14, 0, 2, 2]
seeds = [40901331, 603936023, 588137156, 555485778, 240378915, 86409256, 784030817, 646486771, 321892221, 856537296]
\end{lstlisting}%Call\_18\_Vehicle\_5
\clearpage


\begin{table}[ht]
\centering
\caption{Call\_35\_Vehicle\_7}
\label{tab:call35vehicle7}
\begin{tabular}{|r|r|r|r|r|}
Method & Average objective & Best objective & Improvement (\%) & Running time \\
\hline
Random search & 15924073.22 & 14436028 & 20.19\% & 1.09s\\
Local Search-1-insert & 7495649.90 & 7005670 & 61.90\% & 1.18s\\
Local Search-2-exchange & 8173130.50 & 6859563 & 62.70\% & 1.18s\\
Local Search-3-exchange & 7846618.10 & 6819851 & 62.91\% & 1.23s\\
Simulated Annealing-1-insert & 7649784.40 & 7089562 & 61.44\% & 1.37s\\
Simulated Annealing-2-exchange & 7609380.90 & 6652294 & 63.82\% & 1.23s\\
Simulated Annealing-3-exchange & 7956510.00 & 6870483 & 62.64\% & 1.37s\\
Simulated Annealing & 9439108.00 & 9416775 & 48.79\% & 8.12s\\
\end{tabular}%Call\_35\_Vehicle\_7
\end{table}
\begin{lstlisting}[label={lst:call35vehicle7},caption=Optimal solution call\_35\_vehicle\_7]
sol = [4, 4, 17, 17, 27, 27, 32, 32, 29, 29, 0, 11, 11, 35, 35, 28, 28, 25, 25, 0, 9, 14, 14, 9, 33, 33, 22, 22, 31, 31, 0, 34, 15, 15, 30, 30, 34, 13, 13, 0, 23, 8, 23, 8, 21, 26, 26, 21, 20, 20, 0, 16, 3, 3, 1, 7, 7, 16, 18, 18, 5, 1, 2, 2, 5, 0, 12, 12, 10, 10, 0, 6, 6, 19, 19, 24, 24]
seeds = [786500426, 959532847, 950894748, 230212893, 343586882, 119208331, 613268842, 14355753, 119042580, 701538630]
\end{lstlisting}%Call\_35\_Vehicle\_7
\clearpage


\begin{table}[ht]
\centering
\caption{Call\_80\_Vehicle\_20}
\label{tab:call80vehicle20}
\begin{tabular}{|r|r|r|r|r|}
Method & Average objective & Best objective & Improvement (\%) & Running time \\
\hline
Random search & 39584864.24 & 37697832 & 18.28\% & 2.44s\\
Local Search-1-insert & 17631174.10 & 16254766 & 65.25\% & 3.57s\\
Local Search-2-exchange & 18107201.60 & 17381583 & 62.84\% & 3.00s\\
Local Search-3-exchange & 18725020.20 & 16763949 & 64.16\% & 3.01s\\
Simulated Annealing-1-insert & 17288988.70 & 15640209 & 66.56\% & 3.69s\\
Simulated Annealing-2-exchange & 17611125.00 & 16291560 & 65.17\% & 2.82s\\
Simulated Annealing-3-exchange & 18606671.00 & 16496296 & 64.73\% & 2.76s\\
Simulated Annealing & 17376357.50 & 16938017 & 63.78\% & 34.88s\\
\end{tabular}%Call\_80\_Vehicle\_20
\end{table}
\begin{lstlisting}[label={lst:call80vehicle20},caption=Optimal solution call\_80\_vehicle\_20]
sol = [70, 68, 62, 70, 68, 44, 62, 44, 0, 18, 63, 63, 18, 79, 27, 79, 27, 0, 4, 4, 26, 26, 0, 30, 30, 71, 71, 67, 67, 28, 28, 0, 45, 46, 46, 69, 45, 58, 58, 69, 0, 60, 60, 65, 65, 80, 80, 7, 50, 7, 56, 14, 50, 56, 14, 0, 53, 40, 40, 35, 53, 42, 35, 42, 0, 64, 64, 20, 20, 0, 8, 8, 17, 77, 17, 77, 0, 15, 15, 29, 29, 48, 48, 0, 41, 11, 11, 41, 55, 12, 12, 55, 0, 23, 23, 9, 59, 9, 59, 0, 57, 57, 19, 19, 73, 73, 0, 61, 74, 61, 37, 74, 3, 3, 75, 36, 37, 75, 36, 0, 25, 51, 25, 76, 76, 5, 5, 2, 2, 51, 47, 47, 0, 32, 49, 32, 49, 24, 24, 6, 6, 0, 21, 34, 34, 21, 16, 10, 10, 16,
      0, 54, 1, 54, 1, 0, 39, 39, 78, 78, 72, 72, 33, 33, 0, 43, 43, 52, 52, 0, 13, 13, 22, 22, 31, 31, 38, 38, 66, 66]
seeds = [860860095, 546161525, 245670279, 710560897, 529352907, 636394771, 487810674, 242082782, 570412785, 44805149]
\end{lstlisting}%Call\_80\_Vehicle\_20
\clearpage


\begin{table}[ht]
\centering
\caption{Call\_130\_Vehicle\_40}
\label{tab:call130vehicle40}
\begin{tabular}{|r|r|r|r|r|}
Method & Average objective & Best objective & Improvement (\%) & Running time \\
\hline
Random search & 76627567.00 & 76627567 & 0.00\% & 4.52s\\
Local Search-1-insert & 28309714.70 & 25950511 & 66.13\% & 6.63s\\
Local Search-2-exchange & 28157453.50 & 26420742 & 65.52\% & 4.89s\\
Local Search-3-exchange & 29951029.40 & 28441536 & 62.88\% & 5.52s\\
Simulated Annealing-1-insert & 27700390.50 & 26539400 & 65.37\% & 5.41s\\
Simulated Annealing-2-exchange & 28875623.30 & 27391443 & 64.25\% & 4.97s\\
Simulated Annealing-3-exchange & 30539879.10 & 28741918 & 62.49\% & 4.87s\\
Simulated Annealing & 27819028.00 & 26652657 & 65.22\% & 77.01s\\
\end{tabular}%Call\_130\_Vehicle\_40
\end{table}
\begin{lstlisting}[label={lst:call130vehicle40},caption=Optimal solution call\_130\_vehicle\_40]
sol = [3, 3, 15, 15, 10, 10, 0, 16, 16, 4, 4, 26, 26, 0, 105, 105, 50, 111, 50, 111, 0, 60, 44, 44, 60, 0, 126, 126, 18, 18, 0, 96, 61, 96, 92, 61, 92, 0, 22, 22, 68, 37, 68, 37, 0, 5, 5, 117, 117, 39, 39, 0, 84, 84, 29, 29, 48, 48, 0, 80, 80, 55, 55, 112, 112, 0, 130, 130, 35, 94, 35, 53, 94, 53, 0, 106, 102, 106, 102, 40, 51, 51, 40, 12, 12, 0, 121, 121, 99, 99, 83, 83, 0, 32, 32, 59, 56, 59, 56, 0, 54, 54, 0, 66, 66, 30, 30, 0, 88, 88, 82, 82, 65, 101, 65, 101, 0, 123, 123, 23, 23, 1, 1, 0, 33, 33, 104, 104, 81, 81, 36, 14, 36, 14, 0, 118, 118, 7, 7, 46, 46, 0, 103, 120, 103, 120, 0, 113, 58, 58,
      113, 0, 72, 72, 25, 25, 128, 128, 0, 86, 86, 24, 24, 0, 69, 97, 69, 97, 76, 76, 0, 38, 38, 107, 109, 107, 109, 0, 27, 27, 129, 90, 129, 90, 64, 64, 0, 74, 124, 124, 74, 119, 119, 110, 110, 0, 2, 2, 108, 108, 47, 91, 47, 91, 0, 45, 9, 45, 9, 20, 20, 0, 63, 78, 78, 63, 77, 77, 0, 100, 17, 100, 43, 17, 43, 0, 42, 34, 42, 34, 0, 21, 114, 95, 114, 95, 125, 21, 127, 127, 125, 0, 19, 73, 19, 62, 62, 73, 28, 28, 0, 52, 122, 52, 122, 0, 41, 41, 31, 31, 57, 57, 0, 67, 67, 116, 70, 116, 70, 71, 71, 0, 11, 11, 89, 79, 89, 79, 0, 75, 98, 98, 75, 0, 6, 6, 8, 8, 13, 13, 49, 49, 85, 85, 87, 87, 93, 93, 115, 115]
seeds = [6099242, 216378353, 997469273, 624383985, 929003968, 562647128, 214316800, 49334277, 278401680, 793220627]
\end{lstlisting}%Call\_130\_Vehicle\_40
\clearpage


\begin{table}[ht]
\centering
\caption{Call\_300\_Vehicle\_90}
\label{tab:call300vehicle90}
\begin{tabular}{|r|r|r|r|r|}
Method & Average objective & Best objective & Improvement (\%) & Running time \\
\hline
Random search & 170784643.00 & 170784643 & 0.00\% & 10.66s\\
Local Search-1-insert & 67458950.70 & 65762201 & 61.49\% & 13.55s\\
Local Search-2-exchange & 74664065.40 & 70120922 & 58.94\% & 13.57s\\
Local Search-3-exchange & 81126275.60 & 76977667 & 54.93\% & 14.21s\\
Simulated Annealing-1-insert & 68605376.90 & 66086167 & 61.30\% & 16.79s\\
Simulated Annealing-2-exchange & 73965704.70 & 66507209 & 61.06\% & 18.59s\\
Simulated Annealing-3-exchange & 79768971.10 & 76103903 & 55.44\% & 15.51s\\
\end{tabular}%Call\_300\_Vehicle\_90
\end{table}
\begin{lstlisting}[label={lst:call300vehicle90},caption=Optimal solution call\_300\_vehicle\_90]
sol = [96, 96, 268, 268, 259, 259, 73, 101, 101, 73, 0, 136, 136, 108, 108, 5, 5, 186, 186, 0, 67, 67, 286, 300, 286, 300, 50, 50, 0, 95, 70, 70, 95, 12, 12, 0, 133, 133, 288, 288, 84, 84, 0, 206, 206, 195, 195, 11, 11, 0, 253, 253, 163, 292, 163, 292, 0, 250, 220, 220, 250, 115, 115, 0, 32, 32, 51, 275, 275, 51, 0, 277, 277, 62, 62, 219, 219, 0, 265, 265, 293, 293, 63, 63, 0, 172, 172, 157, 14, 157, 14, 0, 77, 77, 0, 109, 194, 194, 109, 0, 233, 233, 291, 291, 297, 102, 297, 140, 140, 102, 0, 25, 25, 23, 23, 0, 185, 294, 185, 294, 0, 245, 245, 154, 154, 0, 280, 280, 266, 42, 266, 42, 0, 152, 35, 152, 295, 35, 295, 0, 171, 123, 123, 212, 171, 212, 0, 272, 272, 112,
      112, 22, 22, 0, 26, 26, 0, 173, 125, 125, 173, 106, 106, 0, 1, 98, 98, 226, 1, 226, 0, 36, 261, 36, 261, 196, 196, 0, 264, 151, 264, 151, 0, 248, 248, 207, 207, 283, 283, 0, 82, 215, 82, 246, 215, 246, 18, 18, 0, 182, 182, 99, 99, 0, 273, 270, 273, 270, 228, 228, 258, 258, 0, 193, 193, 285, 285, 0, 168, 168, 4, 4, 0, 179, 179, 167, 167, 227, 227, 0, 254, 254, 81, 81, 0, 183, 183, 44, 27, 27, 146, 146, 44, 0, 256, 256, 165, 165, 17, 17, 0, 114, 72, 114, 184, 184, 72, 97, 97, 0, 208, 224, 210, 208, 224, 210, 0, 10, 10, 129, 129, 107, 107, 0, 249, 249, 64, 64, 33, 33, 0, 225, 225, 91, 91, 251, 251, 0, 269, 269, 30, 30, 20, 20, 0, 134, 134, 144, 144, 236, 236, 197, 197, 0, 159, 159, 60, 56, 56, 60, 0, 41, 287, 287, 41, 202, 202, 0, 188, 188, 176, 53, 53, 176, 296, 296, 0, 24, 24, 257, 257, 0, 39, 39, 235, 235, 132, 132, 0, 200, 118, 118, 200, 0, 271, 271, 61, 61, 229, 229, 47, 47, 0, 237, 237, 189, 189, 0, 201, 121, 121, 201, 0, 19, 19, 203, 153, 153, 203, 0, 232, 232, 230, 230, 90, 104, 104, 90, 0, 218, 218, 16, 16, 156, 156, 0, 137, 137, 192, 192, 231, 231, 0, 49, 49, 52, 52, 100, 100, 0, 48, 48, 9, 9, 0, 37, 68, 68, 221, 37, 221, 0, 69, 181, 69, 120, 120, 34, 181, 34, 0, 147, 147, 149, 149, 139, 139, 190, 190, 0, 214, 143, 143, 45, 214, 45, 282, 282, 0, 94, 243, 94, 148, 243, 38, 148, 38, 0, 105, 103, 105, 103, 58, 113, 58, 113, 0, 241, 217, 241, 217, 124, 124, 187, 187, 0, 260, 40, 260, 40, 0, 76, 76, 199, 205, 205, 199, 298, 298, 0, 178, 178, 170, 46, 46, 170, 0, 238, 79, 79, 238, 117, 117, 0, 177, 177, 55, 8, 55, 8, 0, 274, 274, 119, 119, 267, 267, 0, 284, 160, 284, 160, 93, 93, 0, 174, 255, 255, 174, 198, 198, 0, 65, 65, 59, 59, 130, 130, 0, 211, 211, 252, 252, 0, 213, 88, 122, 213, 88, 122, 0, 180, 180, 290, 290, 85, 85, 0, 29, 29, 234, 234, 0, 222, 74, 222, 21, 74, 21, 0, 15, 15, 7, 204, 7, 204, 0, 263, 169, 169, 164, 131, 131, 263, 164, 0, 247, 247, 281, 281, 3, 3, 0, 111, 111, 128, 128, 0, 155, 155, 92, 92, 28, 191, 191, 28, 0, 86, 86, 262, 289, 262, 289, 0, 135, 135, 141, 127, 209, 141, 127, 209, 0, 126, 240, 126, 240, 80, 80, 13, 13, 0, 279, 279, 166, 166, 161, 161, 0, 110, 89, 89, 110, 0, 2, 2, 6, 6, 31, 31, 43, 43, 54, 54, 57, 57, 66, 66, 71, 71, 75, 75, 78, 78, 83, 83, 87, 87, 116, 116, 138, 138, 142, 142, 145, 145, 150, 150, 158, 158, 162, 162, 175, 175, 216, 216, 223, 223, 239, 239, 242, 242, 244, 244, 276, 276, 278, 278, 299, 299]
seeds = [589998094, 96426108, 887972286, 891897509, 195314139, 57672935, 733646253, 892184688, 2275308, 475418836]
\end{lstlisting}%Call\_300\_Vehicle\_90
\clearpage





\end{document}